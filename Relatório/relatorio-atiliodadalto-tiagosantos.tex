\documentclass[12pt, oneside, a4paper, brazil]{abntex2}
% \usepackage[utf8]{inputenc}
\usepackage[brazil]{babel}
% \usepackage[T1]{fontenc}
\usepackage{fontspec}
\usepackage{setspace}
\usepackage{graphicx}
\usepackage{scalefnt}
\usepackage{float}
\usepackage[a4paper, left=3cm, right=2.5cm, top=3cm, bottom = 2.5cm]{geometry}
\usepackage{hyperref}
\usepackage{listings}
\usepackage{color}
\usepackage{titlesec}
%\usepackage[alf]{abntex2cite}
% \usepackage{times}
\usepackage[scaled]{helvet}
\usepackage{fancyhdr}
\usepackage{lmodern}


%%% Definição de variáveis. %%%
% ---
% Informações de dados para CAPA e FOLHA DE ROSTO
% ---
\titulo{Análise de desempenho de estruturas de dados para um indexador de arquivos}
\autor{Atílio Antônio Dadalto \\ Tiago da Cruz Santos}
\local{Vitória}
\data{2019}
\instituicao{%
  Universidade Federal do Espírito Santo
  \par Departamento de Informática}
\tipotrabalho{Relatório}
\preambulo{Relatório apresentado como requisito parcial para aprovação na disciplina de Programação I, pela Universidade Federal do Espírito Santo.}

\newcommand{\versao}{2.0}
\newcommand{\subtitulo}{Anteprojeto}


% Macros específicas do trabalho.
% (*) Inclua aqui termos que são utilizados muitas vezes e que demandam formatação especial.
% Os exemplos abaixo incluem i* (substituindo o asterisco por uma estrela) e Java com TM em superscript.
% Use sempre \xspace para que o LaTeX inclua espaço em branco após a macro somente quando necessário.
\newcommand{\istar}{\textit{i}$^\star$\xspace}
\newcommand{\java}{Java\texttrademark\xspace}
\newcommand{\latex}{\LaTeX\xspace}


%%% Configurações finais de aparência. %%%

% Altera o aspecto da cor azul.
\definecolor{blue}{RGB}{41,5,195}

% Informações do PDF.
\makeatletter
\hypersetup{
	pdftitle={\@title}, 
	pdfauthor={\@author},
	pdfsubject={\imprimirpreambulo},
	pdfcreator={LaTeX with abnTeX2},
	pdfkeywords={abnt}{latex}{abntex}{abntex2}{trabalho acadêmico}, 
	colorlinks=true,				% Colore os links (ao invés de usar caixas).
	linkcolor=blue,					% Cor dos links.
	citecolor=blue,					% Cor dos links na bibliografia.
	filecolor=magenta,				% Cor dos links de arquivo.
	urlcolor=blue,					% Cor das URLs.
	bookmarksdepth=4
}
\makeatother

% Espaçamentos entre linhas e parágrafos.
\setlength{\parindent}{1.3cm}
\setlength{\parskip}{0.2cm}



%%% Páginas iniciais do documento: capa, folha de rosto, ficha, resumo, tabelas, etc. %%%

% Compila o índice. <--- Desnecessário em Plano de Estudo.
%\makeindex

% Inicia o documento.
\begin{document}

% Retira espaço extra obsoleto entre as frases.
\frenchspacing


% Brasão da instituição.
\begin{figure}[h]
  \centering
  \includegraphics[scale=0.055]{figuras/brasao}
\end{figure} 

% Capa do trabalho.
\imprimircapa
\imprimirfolhaderosto

% Lista de silgas.
% (*) Indicar as siglas utilizadas no trabalho como no exemplo abaixo.
%\begin{siglas}
    %\item [UML] Linguagem de Modelagem Unificada, do inglês \textit{Unified Modeling Language}
%\end{siglas}

% Índice de capítulos.
% \tableofcontents*


%%% Início da parte de conteúdo do documento. %%%

% Marca o início dos elementos textuais.
\clearpage
\textual

% Inclusão dos capítulos.
% (*) Para facilitar a organização, os capítulos foram divididos em arquivo separados e colocados dentro da.
% pasta capitulos/. Caso o aluno prefira trabalhar com um só arquivo, basta substituir os comandos \input 
% pelos conteúdos dos arquivos que estão sendo incluídos, excluindo a pasta capitulos/ em seguida.

\chapter{Introdução}
\label{sec-intro}


Apresenta descrição do problema a ser resolvido e visão geral sobre o funcionamento do programa.
% Exemplos de como usar referência: \citeonline{guarino-et-al:hobook09} (in-line) ou~\cite{guarino-et-al:hobook09}.

\chapter{Implementação e funcionamento}\label{cap:implementacao-funcionamento}

Para a implementação do indexador, foram utilizadas diversas estruturas vistas no curso, desde listas encadeadas até tabelas de dispersão. Nas seções que se seguem, é discorrido acerca da composição dessas estruturas. O projeto conta com um total de 7 bibliotecas, sendo 5 focadas nas estruturas e 2 no indexador propriamente dito.

\section{Palavra}
Primeiramente, a estrutura que possivelmente seja a mais importante do projeto, a Palavra, é definida a seguir:
\begin{lstlisting}
/* Estrutura de dados para guardar nomes de arquivos e ocorrências de palavras nestes arquivos */
typedef struct Arq
{
    char *nomeArquivo;
    ListaOcorre *ocorrencias;
    struct Arq *prox;
} Arq;

typedef struct ListaArq
{
    Arq *primeiro;
    Arq *ultimo;
} ListaArq;

/* Estrutura de dados que abstrai uma palavra do texto */
typedef struct Palavra
{
    char *string;
    ListaArq *arquivos;
} Palavra;
\end{lstlisting}

Com ela, podemos armazenar um vetor de caracteres, que será uma palavra do texto, o nome do arquivo em que a palavra se encontra e as posições das ocorrências neste arquivo. Dessa forma, podemos ter conhecimento de ocorrências de uma palavra em um arquivo, além de suas posições, tudo em uma mesma estrutura. O TAD Palavra foi utilizado para todas as cinco estruturas de dados de indexação principais e conta com uma lista encadeada de nomes de arquivos, em que cada célula da lista carrega o nome de arquivo e uma lista encadeada de ocorrências da palavra no arquivo.

\section{Listas encadeadas}
Primeiramente, uma das nossas estruturas mais simples, a lista encadeada, é definida a seguir:
\begin{lstlisting}
// Lista encadeada que abstrai um conjunto de itens
    typedef struct Celula
    {
        Palavra *palavra;
        struct celula *prox;
    } Celula;

    typedef struct Lista
    {
        Celula *primeiro, *ultimo;
    } Lista
\end{lstlisting}

O ponto forte da lista encadeada de indexação seriam suas inserções $O(1)$. Mais tarde, no entanto, veremos que sua inserção é $O(n)$. Além dessa implementação, temos algumas outras variações, criadas para suportar a busca aleatória de palavras, além da utilizada para nomes de arquivos na estrutura Palavra:

\begin{lstlisting}
/* Lista encadeada para armazenar ocorrências */
typedef struct ocorre
{
    int ocorreu;
    struct ocorre *prox;
} CelulaOcorre;

typedef struct ocorrencias
{
    CelulaOcorre *primeiro;
    CelulaOcorre *ultimo;
    int qtd;
} ListaOcorre;

// Lista encadeada para pesquisar palavras aleatoriamente
typedef struct CelulaRandPal
{
    char *string;
    struct pal_rand *prox;
} CelulaRandPal;

typedef struct ListaRandPal
{
    CelulaRandPal *primeiro;
    CelulaRandPal *ultimo;
    int qtd;
} ListaRandPal;
\end{lstlisting}


\section{Árvores Binárias Não Balanceadas}\label{tadArvBin}
\begin{lstlisting}
typedef struct No
{
    Palavra *palavra;
    struct No *esq;
    struct No *dir;
    int altura;
} No;

typedef struct No *ArvBin;
\end{lstlisting}

Para a árvore binária não balanceada, guardamos uma palavra em cada nó, além dos nós esquerda e direita. A raiz é definida como um ponteiro de No.

\section{Árvores Binárias Balanceadas}\label{tadAVL}
Como já visto no curso, a árvore binária balanceada é definida da mesma forma que a não balanceada:
\begin{lstlisting}
typedef struct No *ArvAVL;
\end{lstlisting}

No entanto, as duas se diferem quanto ao balanceamento. Na árvore AVL, ela é constantemente balanceada ao inserir nós com Palavras.

\section{Árvores de prefixo (trie)}\label{tadTrie}
\begin{lstlisting}
typedef struct NoTrie
{
    char letra;
    Palavra *palavra;
    struct NoTrie *filhos[TAM_TRIE]; // ponteiro?
} NoTrie;
\end{lstlisting}

A árvore trie foi implementada armazenando-se uma letra e uma Palavra em cada nó, além de um vetor de nós de filhos do tamanho estipulado \texttt{TAM\_TRIE}, cujo valor é 36 (26 letras do alfabeto + 10 algarismos).


\section{Tabelas de dispersão (tabela hash)}
\begin{lstlisting}
typedef struct tabelahash
{
    ArvAVL *hash[TAM_HASH];
    int colisoes;
    int qtd;
    int *pesos;
} TabelaHash;
\end{lstlisting}
Por fim, temos a tabela hash, com quantidade de colisões, de elementos e o vetor de pesos. Em nossa implementação, optamos por utilizar árvore binárias balanceadas para guardar elementos, visto que é uma estrutura de armazenamento muito eficiente, como notado na elaboração do laboratório 11 da disciplina. \texttt{TAM\_HASH} foi definido como 997, um número primo grande, escolhido após testes com arquivos grandes e pequenos.

\section{Funcionamento}
Após compilar o projeto, o usuário deve fornecer o número de buscas e os arquivos de entrada como argumento no terminal, nesta ordem, como explicado na Introdução. Com o programa em execução, será solicitado ao usuário informar qual estrutura de dados será utilizada para indexação dos arquivos. O usuário pode escolher quais quiser, uma por pedido, ou requisitar que todas sejam utilizadas de uma vez (em sequência). Após carregar os arquivos para a memória e realizar a busca randômica, será solicitado ao usuário que entre com uma palavra para ser buscada na(s) estrutura(s). O programa retornará os dados conforme a existência ou não da palavra no texto e, por fim, informará ao usuário os tempos de inserção, busca aleatória e busca do usuário.

\chapter{Análise  e resultados}
\label{sec-metodo}

\chapter{Metodologia}

Descrever a metodologia dos testes, como variou o tamanho dos arquivos, quantos arquivos foram utilizados, descrição do computador em que foram feitos os testes.

\section{Resultados}

Analisar os tamanhos dos arquivos compactados a partir dos experimentos realizados. Utilizar tabelas e gráficos para ilustrar o desempenho da sua implementação.

Exemplo de utilização de tabelas.  A Tabela ~\ref{tab:cronograma-1} apresenta um cronograma de execução de um PG fictício.


\begin{table}[htb]
	\centering
	\caption{Cronograma de Atividades do primeiro semestre.}
	\label{tab:cronograma-1}
	\resizebox{\columnwidth}{!}{
		\begin{tabular}{c|c|c|c|c|c|c}
			Atividade & Janeiro/99 & Fevereiro/99  & Março/99  & Abril/99 & Maio/99 & Junho/99\\ \hline
			1&     X      &  	  X   	       &  	X	  	   & 		X	&     X   &      X    \\ \hline
			2&            &  	   	     	   &  	  X  	   & 		X	&         &           \\ \hline
			3&            &  		           &  	  X 	   & 		X   &   X     &     X      \\ \hline
			4&            &  			       &  			   & 	        &         &     X      \\ \hline
			5&            &  			       &  			   & 	        &    X    &   X       \\ \hline
			6&            &  			       &  			   & 	        &         &           \\ \hline
			7&            &  			       &  			   & 	        &         &           \\ \hline
		\end{tabular}
	}
\end{table}


A Figura \ref{fig:graf} exemplifica o uso de uma figura gráfica no texto.


   \begin{figure}[!htb]
    \centering
   	\includegraphics[scale=0.90]{figuras/graf-exemplo.png}
   	\caption{Exemplo de inserção de figura}
   	\label{fig:graf}
   \end{figure}

\chapter*{CONCLUSÃO}\label{cap-conclusao}
\addcontentsline{toc}{chapter}{CONCLUSÃO}

Com os dados levantados e a análise feita, podemos então discutir os resultados de uma forma geral. Apesar de uma amostra de 20 iterações não ser ideal, é o suficiente para tomarmos conclusões sobre os desempenhos.

% por conta de sua inserção que percorre a lista toda em busca de um elemento que seja igual ao elemento a ser inserido 
% Mesmo que as inserções das árvores possuam complexidade computacional grande, estas possuem uma buscas mais velozes que a da lista, por exemplo.
Sendo assim, a lista encadeada, como mostrado no Capítulo~\ref{cap-analise-resultado}, é a estrutura mais lenta dentre todas do projeto, possuindo custo $O(n)$ em vez do $O(1)$ usual para inserções de listas encadeadas. Apesar de tudo, a implementação da lista também é extremamente mais simples comparada às implementações das árvores, o que não ajuda apenas a construção inicial de um projeto, mas também sua manutenção.

A árvore binária sem balanceamento... comparar com outras estruturas

A árvore binária balanceada, por sua vez, demonstrou o desempenho mais equilibrado dentre as cinco estruturas; sua inserção é levemente mais lenta que a da lista encadeada, porém a busca consegue ser extremamente mais rápida, com complexidade $O(\log{n})$ no pior caso. Por conta de seu equilíbrio, foi escolhida para acomodar Palavras na tabela hash.

A árvore de prefixos, ou trie, teve o melhor desempenho de todos para buscas... sera mesmo?¿

Por fim, a tabela hash...

Portanto, cabe ao idealizador de um projeto determinar qual estrutura é mais adequada ao problema em questão, analisando, se possível, a razão entre inserção e busca e também se a solução não é complexa demais para o objetivo proposto. Para um projeto diminuto, a lista encadeada poderia ser utilizada sem problemas notáveis de performance, ainda sendo válido destacar sua simples implementação; no caso de um indexador, no entanto, ela representou a pior estrutura. Como visto na seção... a melhor estrutura para inserções, neste caso, foi a .... Já a melhor para buscas foi a ... (trie?). A estrutura X se demonstrou equilibrada quando inserções e buscas são levadas em consideração em conjunto. Para um projeto como este de um indexador de palavras, os autores consideram a estrutura x como a mais capacitada para o empreendimento.

\lipsum[10]


% Finaliza a parte no bookmark do PDF para que se inicie o bookmark na raiz e adiciona espaço de parte no sumário.
\phantompart

% Marca o fim dos elementos textuais.
\postextual

% Referências bibliográficas
\clearpage
\bibliography{bibliografia}

% Fim do documento.
\end{document}
