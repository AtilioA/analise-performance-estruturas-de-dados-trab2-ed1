\chapter{Implementação}
\label{sec-implementacao}


Descrição sobre a implementação dos algoritmos, incluíndo o algoritmo de Huffman. Deve ser detalhada a estrutura de dados utilizada (de preferência com diagramas ilustrativos), o funcionamento dos algoritmos, e decisões tomadas relativas aos detalhes de especificação que porventura estejam omissos no enunciado.


O pacote \texttt{listings}, incluído neste template, permite a inclusão de listagens de código.  A Listagem~\ref{lst-intro-exemplo} mostra um exemplo de listagem com especificação da linguagem utilizada no código. O pacote \texttt{listings} reconhece algumas linguagens\footnote{Veja a lista de linguagens suportadas em \url{http://en.wikibooks.org/wiki/LaTeX/Source\_Code\_Listings\#Supported_languages}.} e faz ``coloração'' de código (na verdade, usa \textbf{negrito} e não cores) de acordo com a linguagem. O parâmetro \texttt{float=htpb} incluído em ambos os exemplos impede que a listagem seja quebrada em diferentes páginas.

Importante notar que não se deve incluir TODO o código do seu trabalho. Inclua apenas trechos que julgue necessário que sejam discutidos no relatório.


\lstinputlisting[label=lst-intro-exemplo, caption=Exemplo de código C especificando linguagem utilizada., language=C, float=htpb]{codigos/lst-exemplo.c}
